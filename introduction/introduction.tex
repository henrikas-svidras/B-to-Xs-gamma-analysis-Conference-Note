\section{Introduction}
%Inclusive analyses of \BtoXsgamma decays enable the determination of Standard Model (SM) observables like $m_{b}$, or can be used as input to SM key observables such as $V_{\mathrm{ub}}$ \cite{Belle-II-physics}.
% Over the past few decades various $B$ physics experiments such as Belle and BaBar, have confirmed many of the Standard Model (SM) predictions in the flavour sector of particle physics. In order to study further this field and investigate potential SM anomalies, the Belle II experiment was started. Belle II is an \epem collider based in Tsukuba, Japan where heavy-to-light $B$ meson decays can be studied. The hadronic \BtoXsgamma decay in particular can currently only be studied at Belle II.

Flavour changing neutral currents (FCNCs) are only allowed in the Standard Model (SM) via loop processes and are therefore highly suppressed \cite{Misiak:2020vlo}. The \BtoXsgamma FCNC decays occur via radiative $b\rightarrow s$ transitions, where $B$ denotes charged and neutral bottom-mesons, and $X_s$ denotes all available final states containing net strangeness. These processes are particularly sensitive to non-SM effects \cite{Misiak:2017bgg}.
In addition, their photon-energy spectrum offers access to various interesting parameters, such as the mass of the $b$ quark and the function describing its motion inside the $B$ meson. \cite{RevModPhys.88.035008, simba}
%or the CKM-matrix elements product $V_{ts}^*V_{tb}$ 
% \cite{RevModPhys.88.035008}. Furthermore, via a global fit, one could also determine the so-called shape function, which describes the motion of the $b$ quark inside the $B$ meson \cite{simba}.
% This function appears in the description of semi-leptonic decays such as $B \rightarrow X_u \ell \nu$ where the CKM element $V_{ub}$ can be measured from. Experimentally, in \BtoXsgamma decays, the low photon energy region is hardly accessible because of the very high background. Therefore, to reduce systematics due to background subtraction, a minimal \Egamma cut has to be imposed. The \Egamma kinematical region has an end-point at $\Egamma \rightarrow m_B/2$.

%Different techniques exist to measure \BtoXsgamma decays such as inclusive and exclusive measurements. In inclusive measurements, no particular $X_s$ hadron ($K^{\pm}$, $K^0$...) is selected but effectively the sum of all possible final states containing an $s$ quark is considered. 
We present an inclusive measurement using \BtoXsgamma decays identified in $\Upsilon(4S) \to B\overline{B}$ events in which the partner $B$ meson is reconstructed in its hadronic decays (hadronic tagging). This approach is complementary to the untagged or lepton-tagged (see e.g., \cite{BaBar:2012fqh}) and sum-of-exclusive (e.g., \cite{Belle:2014nmp}) methods because it has different sources of systematic uncertainty. In addition, tagging provides a purer sample and the kinematic information from the partner-$B$ meson gives direct access to observables in the signal-$B$ meson rest frame. We denote the photon energy in the signal-$B$ meson rest frame as \EB. In this analysis, the photon energy threshold is $1.8~\gev<\EB$. The inclusive analysis does not distinguish between contributions from $b\ra\d\g$ and $b\ra\s\g$ processes, therefore the much smaller $b\ra\d\g$ contribution is subtracted from the final results with a shape determined from simulation.

% The data sample corresponds to 189~\invfb collected between 2019 and 2021 at the \FourS resonance by the Belle II experiment at the SuperKEKB collider.

%Such a measurement has only been published once using parts of the BaBar dataset \cite{BaBarXsGamma}. 
% The predictions for observables from inclusive \BtoXsgamma decays are theoretically clean, and the measurements can provide sensitive probes to physics beyond the SM (e.g. Two-Higgs Doublet model). One of the ways to perform a fully inclusive measurement at Belle II can be done by tagging the hadronic decays originating from the second $B$ meson in an \FourS event. 

% The inclusive \BtoXsgamma branching fraction has already been measured at BaBar with 210 $\mathrm{fb}^{-1}$ of data for photon energies \Egamma above 1.9 GeV in the $B$ rest frame \cite{PhysRevD.77.051103} and at Belle with 605 $\mathrm{fb}^{-1}$ of data and an \Egamma cut of 1.7 GeV \cite{PhysRevLett.103.241801}. 



%\begin{figure}[htbp!]
%    \centering
%    \includegraphics[scale=0.75]{}
%    \caption{: Feynman diagram of $b \rightarrow s \gamma$ at leading order.}
%    \label{fig:bsy_feynman}
%\end{figure}
% \\
% \\
% Need to add references and a Feynman diagram but otherwise I'm good. 

% {\color{red} HS: Drop in references, but Feynman diagrams are probably not needed for the conf note, don't worry. I'll shorten this a bit today but it looks good already :)}
