\section{Belle II detector}

The Belle II \cite{Belle-II:2010dht} detector is designed to reconstruct the final states of electron-positron collisions at center-of-mass energies at or near the \FourS meson mass. The colliding \epem beams are provided by the SuperKEKB collider \cite{AKAI2018188} at KEK in Tsukuba, Japan. The detector has collected physics data since 2019. Belle II consists of several detector subsystems arranged cylindrically around the beam pipe. In the Belle II coordinate system, the $x$ axis is defined to be horizontal and points outside of the accelerator main-rings tunnel, the $y$ axis is vertically upward, and the $z$ axis is defined in the direction of the electron beam. The azimuthal angle, $\phi$, and the polar angle, $\theta$, are defined with respect to the $z$ axis. Three regions in the detector are defined based on $\theta$: forward endcap (\mbox{$12^{\circ}<\theta<31^{\circ}$}), barrel (\mbox{$32^{\circ}<\theta<129^{\circ}$}) and backward endcap (\mbox{$131^{\circ}<\theta<155^{\circ}$}).

The Belle II vertex detector is designed to precisely determine particle decay vertices. It is the innermost subsystem, and consists of a silicon pixel detector and a silicon strip detector. Surrounding the vertexing subsystems is the central drift chamber,
%, which covers the $\theta\in[17-150]^{\circ}$ polar angle region, 
which is used to measure charged-particle trajectories (tracks) to determine their charge and momentum. It also provides important particle-identification information by measuring the specific ionisation of charged tracks. Further particle identification is provided by the time-of-propagation detector and the aerogel ring-imaging Cherenkov detector, which cover, respectively, the barrel and the forward endcap regions of Belle II. Photons and electrons are stopped and their energy deposits (clusters) are read out by the CsI(Tl)-crystal electromagnetic calorimeter. The photon-energy resolution of the ECL is better than 20~\mev for photons above 1~\gev.
% (ECL) which covers the forward (\mbox{$12^{\circ}<\theta<31^{\circ}$}), barrel (\mbox{$32^{\circ}<\theta<129^{\circ}$}) and backward (\mbox{$131^{\circ}<\theta<155^{\circ}$}) regions of the Belle II detector. 
All the inner components are surrounded by a superconducting solenoid, which provides a uniform axial 1.5 T magnetic field. The $K^0_L$ and muon detector, composed of plastic scintillators and resistive-plate chambers, is the outermost subsystem of Belle~II.

\section{Data sets}\label{sec:datasets}


The results presented here use a data sample corresponding to 189~\invfb of integrated luminosity collected at an energy corresponding to the \FourS mass. In addition, an off-resonance data set corresponding to 18~\invfb collected 60~\mev below the \FourS resonance is used to validate the \epem\ra\qqbar simulation. Here $q$ is used to indicate $u,~d,~s$ and $c$ quarks.

The relevant background and signal processes are modeled using large samples simulated through the Monte Carlo~(MC) method corresponding to \mbox{1.6 \invab} of \epem\ra\qqbar events (generated by KKMC~\cite{Ward:2002qq}, interfaced to PYTHIA~\cite{Sjostrand:2007gs}) and  \mbox{\FourS\ra\BzBzb, \BpBm} events (generated by EVTGEN~\cite{Ryd:2005zz}). The detector response is simulated using Geant~4~\cite{Agostinelli:2002hh}.

In addition, inclusive \BtoXsgamma signal distributions are generated using \texttt{BTOXSGAMMA}, the EVTGEN implementation of the Kagan-Neubert model \cite{Kagan:1998ym}, with values of the model parameters taken from Ref. \cite{simba}. The inclusive model, by construction, does not reproduce the resonant structure of the $\b\ra\s\g$ transitions. Therefore, the $\B\ra K^*(892)\gamma$ sample (later denoted as $\B\ra K^*\g$) generated by EVTGEN is also used, as it dominates the higher end of the \Egamma spectrum. The \BtoXsgamma and $\B\ra\Kstar\gamma$  signal simulations are combined using a ``hybrid-model'', inspired by Ref. \cite{hybrid_model}. The full spectrum is modelled by the combination of the two simulated signal samples. A set of hybrid \EB intervals (bins) is defined and the \BtoXsgamma spectrum is scaled in each bin to match the partial branching fraction of the combined \BtoXsgamma and $\B\ra\Kstar\g$ decays with the expected value. The hybrid signal model is used in the selection optimisation, efficiency determination, and unfolding procedure.

% Weights, $w_i$, are calculated for different bins of the \textit{generated} photon energy in the \B-meson rest frame. These weights are defined such that $H_i=R_i+w_iI_i$, with $H_i = I_i$, where $i$ is a given hybrid bin in \EB, $H_i$ is the total number of events per hybrid bin, $R_i$ is the number of resonant $B\ra K^*\gamma$ events per hybrid bin and $I_i$ is the number of inclusive model events in a given hybrid bin. 

All the data sets are analysed using Belle II analysis software framework \cite{Kuhr:2018lps}.
